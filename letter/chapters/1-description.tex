\section{ОПИС МІСЦЯ ПРАКТИКИ}


Кафедра Програмного забезпечення НУ “Львівська політехніка” була створена у серпні 1990 р. В зв’язку з гострим зростанням потреб та відсутністю належних фахівців. Відкриття кафедри зумовили початок підготовки інженерів у цій новій перспективній галузі.

На даний момент кафедра проводить підготовку спеціалістів за різними освітніми рівнями, також на кафедрі проходять різноманітні науково дослідні роботи, працюють різні наукові гуртки.

Навчальний процес на кафедрі забезпечують 9 докторів і 24 кандидати наук. Більшість викладачів поєднують викладацьку діяльність із роботою в провідних ІТ компаніях: Epam, GlobalLogic, Eleks, Lohika, AdvaSoft, Exoft, LinkUp Studio, KindGeek, Dinamica Generale. Завдяки цьому студенти мають змогу проходити навчально-технологічну, переддипломну, дослідницьку практики в цих компаніях набуваючи практичних умінь, навиків та професійного досвіду.
